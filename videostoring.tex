An important role of an experiment in organ-on-a-chip applications, as well as the whole biomedical field, is given by the video analysis. In fact, analyze the video just one time in real-time, during the experiment is running is never sufficient. So, what the system described in this thesis needs to be usable, is a way to memorize the video in order to be used, watched, in a second time.

To fulfill this aim I designed a console application using the \textit{Qt} framework. The motivation that brought me to use this framework is that the same code can run in different platform. In other words, this application can run on \textit{Linux}, \textit{Windows}, and \textit{MacOS}  indistinctly. This is a big advantage, because in a laboratory environment there are many researchers, and it is very easy to encounter different operating systems.

 In (App.\ref{code:video}) the listings of this application are shown. As can be seen, the source codes are divided in such a way to ensure high hierarchical efficiency.
 
 Indeed, the main (List.\ref{code:mainstoring}) of this application just instantiates a \textit{MyTimer} object. This \textit{MyTimer} object (List.\ref{code:timer}) is in charge to generate an interrupt every $10\ sec$, and it starts from its creation. When this interrupt comes, the timer uses the \textit{Downloader} object to check if  a new microscope video has been uploaded, and if so, download it and reset the flag that points up the new video status. The code of \textit{Downloader} is shown in (List.\ref{code:downloader}). The videos are stored inside the directory \textit{C:/Video} and their names correspond to the date and hour of download. 
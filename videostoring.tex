An important role of an experiment in organ-on-a-chip applications, as well as the whole biomedical field, is given by the video analysis. In fact, analyze the video just one time in real-time, during the experiment is running is never sufficient. So, what the system described in this thesis need to be usable is a way to memorize the video in order to be used, watched, in a second time.

To fulfill this aim I designed a console application using the \textit{Qt} framework. The motivation that brought me to use this framework is that the same code can run in different platform. In other words, this application can run on Linux, Windows, and MacOS  indistinctly. This is a big advantage, because in a laboratory environment there are many researcher, and it is very easy to encounter different operating systems.
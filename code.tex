\definecolor{mygreen}{rgb}{0,0.6,0}
\definecolor{mygray}{rgb}{0.5,0.5,0.5}
\definecolor{mymauve}{rgb}{0.58,0,0.82}

\section{Firmware}\label{code:firmware}

\lstinputlisting[frame=single,  basicstyle=\footnotesize, breakatwhitespace=true, 
basicstyle=\scriptsize\ttfamily,,
breaklines=false,
commentstyle=\color{mygreen},
keepspaces=true,
keywordstyle=\color{blue}, 
numbers=left,
numbersep=5pt, 
numberstyle=\tiny\color{mygray}, % the style that is used for the line-numbers
rulecolor=\color{black}, 
showspaces=false,
showstringspaces=false,          % underline spaces within strings only
showtabs=false,                  % show tabs within strings adding particular underscores
stepnumber=2,                    % the step between two line-numbers. If it's 1, each line will be numbered
stringstyle=\color{mymauve},     % string literal style
tabsize=2,                       % sets default tabsize to 2 spaces
language=C++,
caption = ElectrovalvesDriver.cpp,
label = code:Electrovalves]{./code/beaglebone/Electrovalves/GlassInterface.cpp}

\lstinputlisting[frame=single,  basicstyle=\footnotesize, breakatwhitespace=true, 
basicstyle=\scriptsize\ttfamily,,
breaklines=false,
commentstyle=\color{mygreen},
keepspaces=true,
keywordstyle=\color{blue}, 
numbers=left,
numbersep=5pt, 
numberstyle=\tiny\color{mygray}, % the style that is used for the line-numbers
rulecolor=\color{black}, 
showspaces=false,
showstringspaces=false,          % underline spaces within strings only
showtabs=false,                  % show tabs within strings adding particular underscores
stepnumber=2,                    % the step between two line-numbers. If it's 1, each line will be numbered
stringstyle=\color{mymauve},     % string literal style
tabsize=2,                       % sets default tabsize to 2 spaces
language=C++,
caption = sensor\_aquiring.cpp,
label = code:sensors]{./code/beaglebone/Sensors/sensor_acquiring.cpp}

\lstinputlisting[frame=single,  basicstyle=\footnotesize, breakatwhitespace=true, 
basicstyle=\scriptsize\ttfamily,,
breaklines=false,
commentstyle=\color{mygreen},
keepspaces=true,
keywordstyle=\color{blue}, 
numbers=left,
numbersep=5pt, 
numberstyle=\tiny\color{mygray}, % the style that is used for the line-numbers
rulecolor=\color{black}, 
showspaces=false,
showstringspaces=false,          % underline spaces within strings only
showtabs=false,                  % show tabs within strings adding particular underscores
stepnumber=2,                    % the step between two line-numbers. If it's 1, each line will be numbered
stringstyle=\color{mymauve},     % string literal style
tabsize=2,                       % sets default tabsize to 2 spaces
language=bash,
caption = recordVideo,
label = code:recordVideo]{./code/beaglebone/Video/recordVideo}

\lstinputlisting[frame=single,  basicstyle=\footnotesize, breakatwhitespace=true, 
basicstyle=\scriptsize\ttfamily,,
breaklines=false,
commentstyle=\color{mygreen},
keepspaces=true,
keywordstyle=\color{blue}, 
numbers=left,
numbersep=5pt, 
numberstyle=\tiny\color{mygray}, % the style that is used for the line-numbers
rulecolor=\color{black}, 
showspaces=false,
showstringspaces=false,          % underline spaces within strings only
showtabs=false,                  % show tabs within strings adding particular underscores
stepnumber=2,                    % the step between two line-numbers. If it's 1, each line will be numbered
stringstyle=\color{mymauve},     % string literal style
tabsize=2,                       % sets default tabsize to 2 spaces
language=C++,
caption = compute\_beating.cpp,
label = code:beation]{./code/beaglebone/Video/compute_beating.cpp}

\lstinputlisting[frame=single,  basicstyle=\footnotesize, breakatwhitespace=true, 
basicstyle=\scriptsize\ttfamily,,
breaklines=false,
commentstyle=\color{mygreen},
keepspaces=true,
keywordstyle=\color{blue}, 
numbers=left,
numbersep=5pt, 
numberstyle=\tiny\color{mygray}, % the style that is used for the line-numbers
rulecolor=\color{black}, 
showspaces=false,
showstringspaces=false,          % underline spaces within strings only
showtabs=false,                  % show tabs within strings adding particular underscores
stepnumber=2,                    % the step between two line-numbers. If it's 1, each line will be numbered
stringstyle=\color{mymauve},     % string literal style
tabsize=2,                       % sets default tabsize to 2 spaces
language=C++,
caption = Pins Setting,
label = code:pins]{./code/beaglebone/Pins/bo.dts}

\lstinputlisting[frame=single,  basicstyle=\footnotesize, breakatwhitespace=true, 
basicstyle=\scriptsize\ttfamily,,
breaklines=false,
commentstyle=\color{mygreen},
keepspaces=true,
keywordstyle=\color{blue}, 
numbers=left,
numbersep=5pt, 
numberstyle=\tiny\color{mygray}, % the style that is used for the line-numbers
rulecolor=\color{black}, 
showspaces=false,
showstringspaces=false,          % underline spaces within strings only
showtabs=false,                  % show tabs within strings adding particular underscores
stepnumber=2,                    % the step between two line-numbers. If it's 1, each line will be numbered
stringstyle=\color{mymauve},     % string literal style
tabsize=2,                       % sets default tabsize to 2 spaces
language=C++,
caption = GPIO.h,
label = code:gpio]{./code/beaglebone/GPIO.h}

\lstinputlisting[frame=single,  basicstyle=\footnotesize, breakatwhitespace=true, 
basicstyle=\scriptsize\ttfamily,,
breaklines=false,
commentstyle=\color{mygreen},
keepspaces=true,
keywordstyle=\color{blue}, 
numbers=left,
numbersep=5pt, 
numberstyle=\tiny\color{mygray}, % the style that is used for the line-numbers
rulecolor=\color{black}, 
showspaces=false,
showstringspaces=false,          % underline spaces within strings only
showtabs=false,                  % show tabs within strings adding particular underscores
stepnumber=2,                    % the step between two line-numbers. If it's 1, each line will be numbered
stringstyle=\color{mymauve},     % string literal style
tabsize=2,                       % sets default tabsize to 2 spaces
language=C++,
caption = GPIO.cpp]{./code/beaglebone/GPIO.cpp}

\lstinputlisting[frame=single,  basicstyle=\footnotesize, breakatwhitespace=true, 
basicstyle=\scriptsize\ttfamily,,
breaklines=false,
commentstyle=\color{mygreen},
keepspaces=true,
keywordstyle=\color{blue}, 
numbers=left,
numbersep=5pt, 
numberstyle=\tiny\color{mygray}, % the style that is used for the line-numbers
rulecolor=\color{black}, 
showspaces=false,
showstringspaces=false,          % underline spaces within strings only
showtabs=false,                  % show tabs within strings adding particular underscores
stepnumber=2,                    % the step between two line-numbers. If it's 1, each line will be numbered
stringstyle=\color{mymauve},     % string literal style
tabsize=2,                       % sets default tabsize to 2 spaces
language=C++,
caption = HTTP.h,
label = code:http]{./code/beaglebone/http.h}

\lstinputlisting[frame=single,  basicstyle=\footnotesize, breakatwhitespace=true, 
basicstyle=\scriptsize\ttfamily,,
breaklines=false,
commentstyle=\color{mygreen},
keepspaces=true,
keywordstyle=\color{blue}, 
numbers=left,
numbersep=5pt, 
numberstyle=\tiny\color{mygray}, % the style that is used for the line-numbers
rulecolor=\color{black}, 
showspaces=false,
showstringspaces=false,          % underline spaces within strings only
showtabs=false,                  % show tabs within strings adding particular underscores
stepnumber=2,                    % the step between two line-numbers. If it's 1, each line will be numbered
stringstyle=\color{mymauve},     % string literal style
tabsize=2,                       % sets default tabsize to 2 spaces
language=C++,
caption = HTTP.cpp]{./code/beaglebone/http.cpp}

\section{Video Storing Software}\label{code:video}

\lstinputlisting[frame=single,  basicstyle=\footnotesize, breakatwhitespace=true, 
basicstyle=\scriptsize\ttfamily,,
breaklines=false,
commentstyle=\color{mygreen},
keepspaces=true,
keywordstyle=\color{blue}, 
numbers=left,
numbersep=5pt, 
numberstyle=\tiny\color{mygray}, % the style that is used for the line-numbers
rulecolor=\color{black}, 
showspaces=false,
showstringspaces=false,          % underline spaces within strings only
showtabs=false,                  % show tabs within strings adding particular underscores
stepnumber=2,                    % the step between two line-numbers. If it's 1, each line will be numbered
stringstyle=\color{mymauve},     % string literal style
tabsize=2,                       % sets default tabsize to 2 spaces
language=C++,
caption = main.cpp,
label = code:mainstoring]{./code/storing/main.cpp}

\lstinputlisting[frame=single,  basicstyle=\footnotesize, breakatwhitespace=true, 
basicstyle=\scriptsize\ttfamily,,
breaklines=false,
commentstyle=\color{mygreen},
keepspaces=true,
keywordstyle=\color{blue}, 
numbers=left,
numbersep=5pt, 
numberstyle=\tiny\color{mygray}, % the style that is used for the line-numbers
rulecolor=\color{black}, 
showspaces=false,
showstringspaces=false,          % underline spaces within strings only
showtabs=false,                  % show tabs within strings adding particular underscores
stepnumber=2,                    % the step between two line-numbers. If it's 1, each line will be numbered
stringstyle=\color{mymauve},     % string literal style
tabsize=2,                       % sets default tabsize to 2 spaces
language=C++,
caption = VideoStoring.pro,
label = code:pro]{./code/storing/VideoStoring.pro}

\lstinputlisting[frame=single,  basicstyle=\footnotesize, breakatwhitespace=true, 
basicstyle=\scriptsize\ttfamily,,
breaklines=false,
commentstyle=\color{mygreen},
keepspaces=true,
keywordstyle=\color{blue}, 
numbers=left,
numbersep=5pt, 
numberstyle=\tiny\color{mygray}, % the style that is used for the line-numbers
rulecolor=\color{black}, 
showspaces=false,
showstringspaces=false,          % underline spaces within strings only
showtabs=false,                  % show tabs within strings adding particular underscores
stepnumber=2,                    % the step between two line-numbers. If it's 1, each line will be numbered
stringstyle=\color{mymauve},     % string literal style
tabsize=2,                       % sets default tabsize to 2 spaces
language=C++,
caption = mytimer.h,
label = code:timer]{./code/storing/mytimer.h}

\lstinputlisting[frame=single,  basicstyle=\footnotesize, breakatwhitespace=true, 
basicstyle=\scriptsize\ttfamily,,
breaklines=false,
commentstyle=\color{mygreen},
keepspaces=true,
keywordstyle=\color{blue}, 
numbers=left,
numbersep=5pt, 
numberstyle=\tiny\color{mygray}, % the style that is used for the line-numbers
rulecolor=\color{black}, 
showspaces=false,
showstringspaces=false,          % underline spaces within strings only
showtabs=false,                  % show tabs within strings adding particular underscores
stepnumber=2,                    % the step between two line-numbers. If it's 1, each line will be numbered
stringstyle=\color{mymauve},     % string literal style
tabsize=2,                       % sets default tabsize to 2 spaces
language=C++,
caption = mytimer.cpp]{./code/storing/mytimer.cpp}

\lstinputlisting[frame=single,  basicstyle=\footnotesize, breakatwhitespace=true, 
basicstyle=\scriptsize\ttfamily,,
breaklines=false,
commentstyle=\color{mygreen},
keepspaces=true,
keywordstyle=\color{blue}, 
numbers=left,
numbersep=5pt, 
numberstyle=\tiny\color{mygray}, % the style that is used for the line-numbers
rulecolor=\color{black}, 
showspaces=false,
showstringspaces=false,          % underline spaces within strings only
showtabs=false,                  % show tabs within strings adding particular underscores
stepnumber=2,                    % the step between two line-numbers. If it's 1, each line will be numbered
stringstyle=\color{mymauve},     % string literal style
tabsize=2,                       % sets default tabsize to 2 spaces
language=C++,
caption = downloader.h,
label = code:downloader]{./code/storing/downloader.h}

\lstinputlisting[frame=single,  basicstyle=\footnotesize, breakatwhitespace=true, 
basicstyle=\scriptsize\ttfamily,,
breaklines=false,
commentstyle=\color{mygreen},
keepspaces=true,
keywordstyle=\color{blue}, 
numbers=left,
numbersep=5pt, 
numberstyle=\tiny\color{mygray}, % the style that is used for the line-numbers
rulecolor=\color{black}, 
showspaces=false,
showstringspaces=false,          % underline spaces within strings only
showtabs=false,                  % show tabs within strings adding particular underscores
stepnumber=2,                    % the step between two line-numbers. If it's 1, each line will be numbered
stringstyle=\color{mymauve},     % string literal style
tabsize=2,                       % sets default tabsize to 2 spaces
language=C++,
caption = downloader.cpp]{./code/storing/downloader.cpp}

\section{Google App Engine}\label{server}

\lstinputlisting[frame=single,  basicstyle=\footnotesize, breakatwhitespace=true, 
basicstyle=\scriptsize\ttfamily,,
breaklines=false,
commentstyle=\color{mygreen},
keepspaces=true,
keywordstyle=\color{blue}, 
numbers=left,
numbersep=5pt, 
numberstyle=\tiny\color{mygray}, % the style that is used for the line-numbers
rulecolor=\color{black}, 
showspaces=false,
showstringspaces=false,          % underline spaces within strings only
showtabs=false,                  % show tabs within strings adding particular underscores
stepnumber=2,                    % the step between two line-numbers. If it's 1, each line will be numbered
stringstyle=\color{mymauve},     % string literal style
tabsize=2,                       % sets default tabsize to 2 spaces
language=python,
caption = Main Script of Server,
label = code:server]{./code/main.py}

\section{Glassware}

\lstinputlisting[frame=single,  basicstyle=\footnotesize, breakatwhitespace=true, 
basicstyle=\scriptsize\ttfamily,,
breaklines=false,
commentstyle=\color{mygreen},
keepspaces=true,
keywordstyle=\color{blue}, 
numbers=left,
numbersep=5pt, 
numberstyle=\tiny\color{mygray}, % the style that is used for the line-numbers
rulecolor=\color{black}, 
showspaces=false,
showstringspaces=false,          % underline spaces within strings only
showtabs=false,                  % show tabs within strings adding particular underscores
stepnumber=2,                    % the step between two line-numbers. If it's 1, each line will be numbered
stringstyle=\color{mymauve},     % string literal style
tabsize=2,                       % sets default tabsize to 2 spaces
language=Java,
caption = MainService.java,
label = code:MainService]{./code/glass/MainService.java}


\lstinputlisting[frame=single,  basicstyle=\footnotesize, breakatwhitespace=true, 
basicstyle=\scriptsize\ttfamily,,
breaklines=false,
commentstyle=\color{mygreen},
keepspaces=true,
keywordstyle=\color{blue}, 
numbers=left,
numbersep=5pt, 
numberstyle=\tiny\color{mygray}, % the style that is used for the line-numbers
rulecolor=\color{black}, 
showspaces=false,
showstringspaces=false,          % underline spaces within strings only
showtabs=false,                  % show tabs within strings adding particular underscores
stepnumber=2,                    % the step between two line-numbers. If it's 1, each line will be numbered
stringstyle=\color{mymauve},     % string literal style
tabsize=2,                       % sets default tabsize to 2 spaces
language=Java,
caption = AppDrawer.java,
label = code:AppDrawer]{./code/glass/AppDrawer.java}

\lstinputlisting[frame=single,  basicstyle=\footnotesize, breakatwhitespace=true, 
basicstyle=\scriptsize\ttfamily,,
breaklines=false,
commentstyle=\color{mygreen},
keepspaces=true,
keywordstyle=\color{blue}, 
numbers=left,
numbersep=5pt, 
numberstyle=\tiny\color{mygray}, % the style that is used for the line-numbers
rulecolor=\color{black}, 
showspaces=false,
showstringspaces=false,          % underline spaces within strings only
showtabs=false,                  % show tabs within strings adding particular underscores
stepnumber=2,                    % the step between two line-numbers. If it's 1, each line will be numbered
stringstyle=\color{mymauve},     % string literal style
tabsize=2,                       % sets default tabsize to 2 spaces
language=Java,
caption = MainView.java,
label = code:MainView]{./code/glass/MainView.java}

\lstinputlisting[frame=single,  basicstyle=\footnotesize, breakatwhitespace=true, 
basicstyle=\scriptsize\ttfamily,,
breaklines=false,
commentstyle=\color{mygreen},
keepspaces=true,
keywordstyle=\color{blue}, 
numbers=left,
numbersep=5pt, 
numberstyle=\tiny\color{mygray}, % the style that is used for the line-numbers
rulecolor=\color{black}, 
showspaces=false,
showstringspaces=false,          % underline spaces within strings only
showtabs=false,                  % show tabs within strings adding particular underscores
stepnumber=2,                    % the step between two line-numbers. If it's 1, each line will be numbered
stringstyle=\color{mymauve},     % string literal style
tabsize=2,                       % sets default tabsize to 2 spaces
language=Java,
caption = BeatingView.java,
label = code:BeatingView]{./code/glass/BeatingView.java}

\lstinputlisting[frame=single,  basicstyle=\footnotesize, breakatwhitespace=true, 
basicstyle=\scriptsize\ttfamily,,
breaklines=false,
commentstyle=\color{mygreen},
keepspaces=true,
keywordstyle=\color{blue}, 
numbers=left,
numbersep=5pt, 
numberstyle=\tiny\color{mygray}, % the style that is used for the line-numbers
rulecolor=\color{black}, 
showspaces=false,
showstringspaces=false,          % underline spaces within strings only
showtabs=false,                  % show tabs within strings adding particular underscores
stepnumber=2,                    % the step between two line-numbers. If it's 1, each line will be numbered
stringstyle=\color{mymauve},     % string literal style
tabsize=2,                       % sets default tabsize to 2 spaces
language=Java,
caption = PHViewer.java,
label = code:PHViewer]{./code/glass/PHViewer.java}

\lstinputlisting[frame=single,  basicstyle=\footnotesize, breakatwhitespace=true, 
basicstyle=\scriptsize\ttfamily,,
breaklines=false,
commentstyle=\color{mygreen},
keepspaces=true,
keywordstyle=\color{blue}, 
numbers=left,
numbersep=5pt, 
numberstyle=\tiny\color{mygray}, % the style that is used for the line-numbers
rulecolor=\color{black}, 
showspaces=false,
showstringspaces=false,          % underline spaces within strings only
showtabs=false,                  % show tabs within strings adding particular underscores
stepnumber=2,                    % the step between two line-numbers. If it's 1, each line will be numbered
stringstyle=\color{mymauve},     % string literal style
tabsize=2,                       % sets default tabsize to 2 spaces
language=Java,
caption = TemperatureView.java,
label = code:TemperatureView]{./code/glass/TemperatureView.java}

\lstinputlisting[frame=single,  basicstyle=\footnotesize, breakatwhitespace=true, 
basicstyle=\scriptsize\ttfamily,,
breaklines=false,
commentstyle=\color{mygreen},
keepspaces=true,
keywordstyle=\color{blue}, 
numbers=left,
numbersep=5pt, 
numberstyle=\tiny\color{mygray}, % the style that is used for the line-numbers
rulecolor=\color{black}, 
showspaces=false,
showstringspaces=false,          % underline spaces within strings only
showtabs=false,                  % show tabs within strings adding particular underscores
stepnumber=2,                    % the step between two line-numbers. If it's 1, each line will be numbered
stringstyle=\color{mymauve},     % string literal style
tabsize=2,                       % sets default tabsize to 2 spaces
language=Java,
caption = VideoPlayerActivity.java,
label = code:VideoPlayerActivity]{./code/glass/VideoPlayerActivity.java}

\lstinputlisting[frame=single,  basicstyle=\footnotesize, breakatwhitespace=true, 
basicstyle=\scriptsize\ttfamily,,
breaklines=false,
commentstyle=\color{mygreen},
keepspaces=true,
keywordstyle=\color{blue}, 
numbers=left,
numbersep=5pt, 
numberstyle=\tiny\color{mygray}, % the style that is used for the line-numbers
rulecolor=\color{black}, 
showspaces=false,
showstringspaces=false,          % underline spaces within strings only
showtabs=false,                  % show tabs within strings adding particular underscores
stepnumber=2,                    % the step between two line-numbers. If it's 1, each line will be numbered
stringstyle=\color{mymauve},     % string literal style
tabsize=2,                       % sets default tabsize to 2 spaces
language=Java,
caption = MenuActivity.java,
label = code:MenuActivity]{./code/glass/MenuActivity.java}

\lstinputlisting[frame=single,  basicstyle=\footnotesize, breakatwhitespace=true, 
basicstyle=\scriptsize\ttfamily,,
breaklines=false,
commentstyle=\color{mygreen},
keepspaces=true,
keywordstyle=\color{blue}, 
numbers=left,
numbersep=5pt, 
numberstyle=\tiny\color{mygray}, % the style that is used for the line-numbers
rulecolor=\color{black}, 
showspaces=false,
showstringspaces=false,          % underline spaces within strings only
showtabs=false,                  % show tabs within strings adding particular underscores
stepnumber=2,                    % the step between two line-numbers. If it's 1, each line will be numbered
stringstyle=\color{mymauve},     % string literal style
tabsize=2,                       % sets default tabsize to 2 spaces
language=Java,
caption = AppManager.java,
label = code:AppManager]{./code/glass/AppManager.java}

\lstinputlisting[frame=single,  basicstyle=\footnotesize, breakatwhitespace=true, 
basicstyle=\scriptsize\ttfamily,,
breaklines=false,
commentstyle=\color{mygreen},
keepspaces=true,
keywordstyle=\color{blue}, 
numbers=left,
numbersep=5pt, 
numberstyle=\tiny\color{mygray}, % the style that is used for the line-numbers
rulecolor=\color{black}, 
showspaces=false,
showstringspaces=false,          % underline spaces within strings only
showtabs=false,                  % show tabs within strings adding particular underscores
stepnumber=2,                    % the step between two line-numbers. If it's 1, each line will be numbered
stringstyle=\color{mymauve},     % string literal style
tabsize=2,                       % sets default tabsize to 2 spaces
language=Java,
caption = DataPoint.java,
label = code:DataPoint]{./code/glass/DataPoint.java}





\begin{figure}[h]
	\subfloat{%First sub-figure\label{subfig-1:dummy}]First sub-figure\label{subfig-2:dummy}]First sub-figure\label{subfig-2:dummy}]First sub-figure\label{subfig-2:dummy}]{%
		\includegraphics[width=.25\textwidth]{./firstpage/HST}
	}
\end{figure}


The present scripture represent the master thesis of Fabio Busignani, and it has been carried out at \href{http://www.tissueeng.net/lab/}{\textit{Khademhosseini Lab}} (Cambdridge, MA, USA), \href{https://hst.mit.edu/}{Harvard-MIT Health Science and Technology}, \href{http://www.brighamandwomens.org/}{Brigham and Women's Hospital} under the supervision of professor Ali Khademhosseini and Ph.D. Yu Shrike Zhang.\\
The design which is going to be described, has been inserted inside the context of a five years project (\textit{\textbf{XCEL}} grant), sponsored by the U.S. Defense Threat Reduction Agency (\textit{DTRA}).\\

The aim of \textit{XCEL} is to develop a \textit{Body-On-A-Chip} microfluidic platform that is able to simulate multi-tissue interactions under physiological fluid flow conditions.\\

This master thesis will focus on designing a custom user interface on \textit{Google Glass} for simultaneous recording of biosensing data such as temperature, pH, and microscopy images/videos as well as remote control of microfluidic valves and devices.  The project involves all the hierarchical layers, starting from the physical one with the circuit in charge to acquire data from bio-sensors and drive the valves, up to the glasswear\footnote{Google Glass Application}.\\

In the Introduction chapter, the main keys of the project are presented in detail as well as the final result from a user point of view.\\
After that, a detailed description of each abstraction level which goes to build the entire systems is shown: starting from the bottom (Hardware) reaching the top (Google Glass Application) passing through the Firmware, that runs in an embedded \textit{Linux} platform, and the Software, present on the \textit{PC} and the \textit{Google App Engine}.\\
The  Experiments and Conclusion chapter shows the obtained results with different experiments. The thesis ends discussing the limitations and improvements of the system.
